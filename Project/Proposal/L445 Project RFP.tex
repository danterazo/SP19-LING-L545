%%%%%%%%%%%%%%%%%%%%%%%%%%%%%%%%%%%%%%%%%
% Stylish Article
% LaTeX Template
% Version 2.1 (1/10/15)
%
% This template has been downloaded from:
% http://www.LaTeXTemplates.com
%
% Original author:
% Mathias Legrand (legrand.mathias@gmail.com) 
% With extensive modifications by:
% Vel (vel@latextemplates.com)
%
% License:
% CC BY-NC-SA 3.0 (http://creativecommons.org/licenses/by-nc-sa/3.0/)
%
%%%%%%%%%%%%%%%%%%%%%%%%%%%%%%%%%%%%%%%%%
% Document provided by Dr. Hasan Kurban of Indiana University Bloomington

%-------------------------------------------------------------------------------------
%	PACKAGES AND OTHER DOCUMENT CONFIGURATIONS
%-------------------------------------------------------------------------------------

\documentclass[fleqn,10pt]{SelfArx} % Document font size and equations flushed left

\usepackage[english]{babel} % Specify a different language here - english by default

\usepackage{lipsum} % Required to insert dummy text. To be removed otherwise

\usepackage{varwidth}

%-------------------------------------------------------------------------------------
%	COLUMNS
%-------------------------------------------------------------------------------------

\setlength{\columnsep}{0.55cm} % Distance between the two columns of text
\setlength{\fboxrule}{0.75pt} % Width of the border around the abstract

%-------------------------------------------------------------------------------------
%	COLORS
%-------------------------------------------------------------------------------------

\definecolor{color1}{RGB}{0,0,90} % Color of the article title and sections
\definecolor{color2}{RGB}{0,20,20} % Color of the boxes behind the abstract and headings

%-------------------------------------------------------------------------------------
%	HYPERLINKS
%-------------------------------------------------------------------------------------

\usepackage{hyperref} % Required for hyperlinks
\hypersetup{hidelinks,colorlinks,breaklinks=true,urlcolor=color2,citecolor=color1,linkcolor=color1,bookmarksopen=false,pdftitle={Title},pdfauthor={Author}}

%-------------------------------------------------------------------------------------
%	ARTICLE INFORMATION
%-------------------------------------------------------------------------------------

\JournalInfo{LING-L 445: Computation and Linguistic Analysis 2019} % Journal information
\Archive{} % Additional notes (e.g. copyright, DOI, review/research article)

\PaperTitle{LING-L 445 Course Project RFP} % Article title	

\Authors{Dante Razo\textsuperscript{1}} % Authors
\affiliation{\textsuperscript{1}\textit{Computer Science; School of Informatics, Computing and Engineering, Indiana University, Bloomington, IN, USA}} % Author affiliation

\Keywords{chuvash --- deep neural networks --- linguistics --- speech synthesis --- tts} % Keywords - if you don't want any simply remove all the text between the curly brackets
\newcommand{\keywordname}{Keywords} % Defines the keywords heading name

%------------------------------------------------------------------------
%	ABSTRACT
%------------------------------------------------------------------------
\Abstract{Chuvash is a minority language spoken by roughly one million people in European Russia. This project aims to train speech synthesizers with neural networks to reproduce intelligible Chuvash from text. The synthesizers used are Ossian, Mozilla TTS, and Mozilla LPCNet. At the minimum, the performance of the three will be compared. The Expected Product involves cleaning the Chuvash data to produce better results with one of the three synthesizers. If time permits, the best-performing synthesizer will be tweaked to produce even more accurate results. Data will come from reliable sources such as the Apertium project. The end goal is to train a neural network that mimics Chuvash speech to some degree, then improve on it as much as time allows.}

%------------------------------------------------------------------------

\begin{document}
\flushbottom % Makes all text pages the same height
\maketitle % Print the title and abstract box
\tableofcontents % Print the contents section
\thispagestyle{empty} % Removes page numbering from the first page

%------------------------------------------------------------------------
% Dante's Notes
%------------------------------------------------------------------------
% Guidelines: http://cl.indiana.edu/~ftyers/courses/2019/L-545/projects/rfp.html
	
%Common voice - lots of Langs don't have speech synthesis
%Clean data before using
%Same speaker (ID)

%Could do Chuvash

%Project:
%MVP: compare state of the art
%medium: extend them or (easier) clean data better

%------------------------------------------------------------------------
% SECTION 1: Intro
%------------------------------------------------------------------------
\section{Introduction}
The Chuvash language is spoken by roughly one million people in European Russia. Despite the large number of speakers, it is considered a minority language. This project aims to train popular speech synthesizers to produce intelligible Chuvash from written samples.

%------------------------------------------------------------------------
% SECTION 2: Goals
%------------------------------------------------------------------------
\section{Proposed Goals}
\subsection{Minimum Viable Product (MVP)}
My MVP is to compare the performance of existing speech synthesizers with the Chuvash language. I plan on using the following solutions:
\begin{enumerate}
	\item Ossian
	\item Mozilla TTS
	\item Mozilla LPCNet
\end{enumerate}

The Ossian library will use deep neural networks (DNNs) trained by the Merlin toolkit as models. Mozilla TTS is a deep learning speech synthesis engine that accepts preprocessed datasets. Mozilla's LPCNet requires at least an hour of speech data; I'll be using Francis M. Tyers' \texttt{Turkic\_TTS} repository to train it. It remains to be seen whether this will be enough data to properly train LPCNet.

In order to understand the subject matter, I will work on memorizing the Cyrillic script and Chuvash's additional four letters. This will help me sound out text and isolated symbols (disregarding irregular phonological features such as palatalization). Almost a third of the characters used in Chuvash are reserved for Russian loanwords, but they are worth learning in case they appear in the data.

\subsection{Expected Product (EP)}
I will explore new methods of preprocessing the data in hopes of improving synthesizer accuracy. Due to the fact that data has to be processed differently to work for each synthesizer, I will likely choose one to focus on for the duration of the project.

\subsection{High-achievement Product (HAP)}
Comparing the different speech synthesizers will give me insight into their strengths and weaknesses. With this knowledge, I will tweak them (or my preferred synthesizer) in an effort to improve their effectiveness. If I see significant improvement in accuracy, I will consider reaching out to the original developers and discussing how to incorporate my changes with their source code.


%------------------------------------------------------------------------
% SECTION 3: Project Requirements
%------------------------------------------------------------------------
\section{Requirements}
\subsection{MVP Goals}
\subsubsection{Subgoal 1}
Configure all three speech synthesizers as described in their tutorials, and confirm that they work.
\subsubsection{Subgoal 2}
Find suitable Chuvash data for training and process it for use with all synthesizers
\subsubsection{Subgoal 3}
Train synthesizers on Chuvash data; report results.

\subsection{EP Goals}
\subsubsection{Subgoal 1}
Clean and preprocess Chuvash data for cleaner input.
\subsubsection{Subgoal 2}
Employ different preprocessing techniques until model accuracy is significantly higher than it was before.

\subsection{HAP Goals}
\subsubsection{Subgoal 1}
Understand the inner workings and parameters of the chosen speech synthesizer.
\subsubsection{Subgoal 2}
Tweak the synthesizer to achieve higher accuracy with Chuvash data.
\subsubsection{Subgoal 3 (Stretch Goal)}
Coordinate with developers to incorporate my changes into the official release/repository of the synthesizers.

%------------------------------------------------------------------------
% SECTION 4: Timelines
%------------------------------------------------------------------------
\section{Timelines} % weekly basis; subject to change
\subsection{MVP Timeline}
\begin{enumerate}
	\item Subgoal 1: \textit{March 12, 2019}
	\item Subgoal 2: \textit{March 15, 2019}
	\item Subgoal 3: \textit{March 17, 2019}
\end{enumerate}

\subsection{EP Timeline}
\begin{enumerate}
	\item Subgoal 1: \textit{March 24, 2019}
	\item Subgoal 2: \textit{March 31, 2019}
\end{enumerate}

\subsection{HAP Timeline}
\begin{enumerate}
	\item Subgoal 1: \textit{April 7, 2019}
	\item Subgoal 2: \textit{April 13, 2019}
	\item Subgoal 3: \textit{N/A (Stretch Goal)}
\end{enumerate}

\subsection{Project Timeline}
\begin{enumerate}
	\item Paper: by \textit{April 12, 2019}
	\item Project: by \textit{April 15, 2019}
\end{enumerate}

%------------------------------------------------------------------------
% SECTION 5: Data Policy
%------------------------------------------------------------------------
\section{Data Policy}
Despite the scarcity of readily-available Chuvash datasets, I aim to only use reliable data. Currently, I'm looking to use Francis Tyer's \texttt{Turkic\_TTS} repository, as well as the \texttt{apertium-chv} transducer for Chuvash. It's unclear whether a transducer is necessary for training the synthesizers.

The synthesizers I'll be using can be found on their respective GitHub repositories (listed below). I will follow typical software development procedures such as making regular commits, and committing every change to make backtracking easy. I'm admittedly unfamiliar with how to create branches in Git, so I'll be committing to \texttt{master}.

My final report will include easily-reproducible instructions so that other scientists will be able to achieve the same (or similar) results. In the event that my adjustments to any synthesizer improves performance, I will get in touch with the original developer to discuss how to merge our code. This might be as simple as making a pull request.

\subsection{Data Sources}
\begin{enumerate}
	\item \href{https://github.com/ftyers/Turkic_TTS}{\underline{Turkic\_TTS} by Francis M. Tyers (ftyers)}
	\item \href{https://github.com/apertium/apertium-chv}{\underline{apertium-chv} (GPL-3.0) by Apertium}
\end{enumerate}

\subsection{Repositories}
\begin{enumerate}
	\item \href{https://github.com/mozilla/TTS}{\underline{Mozilla TTS} (MPL-2.0) by Mozilla}
	\item \href{https://github.com/CSTR-Edinburgh/Ossian}{\underline{Ossian} (Apache-2.0) by CSTR-Edinburgh}
	\item \href{https://github.com/mozilla/LPCNet}{\underline{Mozilla LPCNet} (BSD-3-Clause) by Mozilla}
\end{enumerate}


%------------------------------------------------------------------------
% REFERENCES
%------------------------------------------------------------------------
%\section{References}
%\phantomsection
\bibliographystyle{unsrt}
\bibliography{sample}

\begin{enumerate}
	\item \href{http://jrmeyer.github.io/tts/2017/09/15/Ossian-Merlin-demo.html}{\underline{Create New Voice with Ossian \& Merlin} by Josh Meyer}
	\item \href{http://jrmeyer.github.io/tts/2016/12/09/tts-workshop.html}{\underline{Let's make a Chuvash voice!} by Josh Meyer}
	\item \href{https://people.xiph.org/~jm/demo/lpcnet/}{\underline{Mozilla LPCNet Demo} by Jean-Marc Valin}
\end{enumerate}

%----------------------------------------------------------------------------------------
\end{document}